
% =====================
\section{Foundations}
\label{sec:foundations}

\subsection{Ur-Fabric and Roles}
We posit a minimal role calculus with observer $O$, generator $G$, counters $C$ and $C^\ast$, and a noise--structure axis $(N\!\leftrightarrow\!S)$ mediated by thresholds.
Here, $C$ counts fast, local events ("sparks"), while $C^\ast$ accumulates slow, integrated return channels ("glow").
This division is operational: $C$ is observable through short-window statistics; $C^\ast$ emerges as long-horizon drift, hysteresis, or cross-scale return.

\subsection{Chiasmus Pattern}
Across levels (micro $\leftrightarrow$ macro; human $\leftrightarrow$ AI; fabric $\leftrightarrow$ cosmos) we assume a chiasmus: dual strands that invert roles under coupling.
Formally, pairs $(X,Y)$ develop coordination in a band, while their complements re-balance via $C^\ast$.
This yields a characteristic swap of agency over time: what is counted by $C$ in one phase becomes carrier for $C^\ast$ in the next.

\subsection{Benchmarks as Physics Anchors}
We introduce falsifiable anchors:
(i) T2 cross-coherence (band-specific synchrony),
(ii) T3 hysteresis (path dependence of the synchrony under parameter sweep).
Both tests operate on the same signals and differ only by the perturbation protocol, making the suite compact and reproducible.

\subsection{Reset, Stagnation, Evolution}
Resets (hard phase randomization) eliminate apparent order; stagnation keeps statistics stationary.
Evolution requires a nonzero $C^\ast$ channel so that structure returns across windows.
Empirically, T2 rejects phase-only nulls (structure exists) and remains detectable under the conservative both-null in a minority of seeds (structure is subtle but real).
T3 detects nonlinearity/memory via a nonzero loop area, consistent with $C^\ast>0$.
